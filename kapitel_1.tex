
\chapter{Mathematische\ Modellierung\ des\ Zufalls}
\section{Zufallsexperimente}
\subsection{Das Würfeln}
\Vorlesung{14.4.2010}
$\Omega = \lbrace 1, ... , n  \rbrace , n \in \mathbb{N}$ Wir möchten zufällig genau eine Zahl aus $\Omega$ ziehen. Eine Möglichkeit: n-seitiger Würfel\newline
Ansatz: $Pr[i] = \frac{1}{n}  \forall i \in \Omega$\newline
$Pr \mathrel{\widehat{=}} ``Probability''$\newline
Sei $A=\lbrace a_1, ... , a_k \rbrace \subset \Omega$, dann ist $Pr[A]=\frac{\left|A\right|}{n} = \frac{k}{n}$ die Wahrscheinlichkeit, dass $a_1, ..., a_{k-1}$ oder $a_{k}$ ausgewählt werden.\newline
A nennt man \Begriff{Ergeignis}\newline
Wenn alle $Pr[i]$ gleich sind so spricht man von einer \Begriff{Gleichverteilung}.\newline
Bei Spielen: n=6
Es herrscht Unabhängigkeit der Würfe, d.h. Ergebnisse beeinflussen sich nicht.

\subsection{Das Lottospiel}
Es werden 6 Zahlen aus 49 gezogen, sagen wir $a_1, .. a_6$. Wir nehmen an, dass wir diese schon geordnet haben: $a_1 < ... < a_6$. Eine Ziehung ist ein Vektor $(a_1,...,a_6)$ mit $a_1 < ... < a_6$. Ergebnisse sind diese Vektoren. Man fasst die Ergebnisse zu einem \Begriff{Grundraum} zusammen, den wir üblicherweise $\Omega$ nennen.
$$ \Omega = \lbrace \lbrace a_1,...,a_6 \rbrace \mid a_i \in \lbrace 1,...,49 \rbrace \forall i=1...6  \rbrace$$
$$ \left|\Omega\right| = \binom{49}{6} = 13983816$$%
Wie hoch ist die Wahrscheinlichkeit, dass ein Tipp 6 Richtige hat? Allgemeiner: k Richtige? %
Welche $\lbrace a_1,...,a_6 \rbrace \in \Omega$ haben $k$ Stellen gemeinsam mit dem Tipp? %
Die bezeichnen wir als \Begriff{günstige\ Ereignisse}.\newpage%

$$A = \lbrace \lbrace a_1,...,a_6 \rbrace \in \Omega \mid \left| \lbrace a_1,...,a_6 \rbrace \bigcup \lbrace b_1,...,b_6 \rbrace \right| = k \rbrace$$%

Wenn ein Element aus $A$ gezogen wird haben wir $k$ Richtige.%
$$ Pr[k \textrm{Richtige}]=\frac{\left| A \right|}{\left| \Omega \right|} = \frac{\binom{6}{k} \cdot \binom{43}{6-k}}{\binom{49}{6}}$$
$$\Rightarrow Pr[6 \textrm{Richtige}]= \frac{\binom{6}{6} \binom{43}{0}}{\binom{49}{6}}=\binom{1}{49}$$
\Vorlesung{19.4.2010}
\subsection{Das Geburtstagsproblem}
Wie hoch ist die Wahrscheinlichkeit, dass zwei Personen in einer Gruppe von $m$ Personen am gleichen Tag Geburtstag haben?
z.B.: $m=50$
% Abb 1
\subsection{Das Ziegenproblem}
Bei einem Fernsehquiz gibt es drei Türen, wobei hinter zwei Türen eine Ziege steht und hinter der dritten Tür ein Hauptgewinn. %
Der Showmaster fragt den Kandidaten, welche Tür er öffnen soll. Nachdem der Kandidat einen Tipp abgegeben hat, bietet der Showmaster seine Hilfe an: %
Er schlägt vor eine Tür zu öffnen und danach soll der Kandidat die Möglichkeit haben seinen Tipp zu verändern.%
Würden Sie sein Angebot annehmen?
% Abb 2
\begin{quote}
 In diesem Fall ist es nicht so einfach, da die günstigen Ergeignisse nicht unabhängig sind.
\end{quote}
\section{Algebra und Maße}
Die axiomatische Fundierung der Wahrscheinlichkeitsrechnung wurder von Kolmogoroff Anfang des 20. Jahrhunderts gegeben. Erst dadurch wurde Wahrscheinlichkeitsrechnung eine wissenschaftliche und quantitative Disziplin.
\subsection{Die Axiome von Komolgoroff}
\begin{itemize}
 \item Ergebnisse eines Zufallsexperimentes fassen wir in einer Menge $\Omega$ zusammen.
 \item \Begriff{Ereignisse} sind gewisse Teilmengen von $\Omega$.
 \item Wahrscheinlichkeiten werden auf Ereignisse bezogen und sind Zahlen zwischen 0 und 1. Eine Zahl $p \in [0,1]$,z.B. $p=\frac{3}{4}$ besagt dass das entsprechende Ereigniss in $\frac{3}{4}$ der Zufallsexperimente auftrit, oder die Wahrscheinlichkeit ist 75 Prozent.

$p=0$: Das Ereignis tritt nicht auf.

$p=1$: Das Ereignis tritt stets auf.
\end{itemize}
Beispiel: Wir werfen eine faire Münze. Kopf und Zahl kommen mit der gleichen Wahrscheinlichkeit.
$$ \textrm{Menge } \Omega = \{\textrm{Kopf},\textrm{Zahl}\} = \{0,1\} $$
$$ \textrm{Ereignisse} = \{\{0\},\{1\},\emptyset,\{0,1\}\} = \mathcal P(\Omega) = \textrm{Potenzmenge von } \Omega $$
$$ P(\{0\})=\frac{1}{2}=P(\{1\}), P(\emptyset)=0 , P(\{0,1\})=1$$
\subsection{``Erweiterung'' der Kolmogoroff-Axiome}

Wie muss die Menge der Ereignisse aussehen?
\begin{itemize}
 \item $A$ Ereignis mit Wahrscheinlichkeit $p$. Wir möchten das Gegenereignis (``$A$ tritt nicht auf'' = $A^c$) ebenfalls als Ereignis ansehen mit der Wahrscheinlichkeit $1-p$.
 \item Ereignisse $A$,$B$ mit den Wahrscheinlichkeiten $p$ und $q$ . Was ist das Ereignis ``$A$ oder $B$'' bzw. ?

 ``$A$ oder $B$'' = $A \cup B$

 ``$A$ und $B$'' = $A \cap B$

 Mithin sollten auch diese Ereignisse in der Ereignismenge drin sein.
 \item Was ist $P(A \cup B)$?

Situation: $A \cap B = \emptyset$ dann ist $P(A \cup B) = P(A) + P(B)$

\end{itemize}
\subsection{$\sigma$-Algebra}
Definition 1.1 Es sei $\Omega$ eine Menge und $\Sigma \subseteq \mathcal P(\Omega)$. $\Sigma$ heißt $\sigma$-Algebraüber $\Omega$ fallse folgende Eigenschaften gelten:
\begin{quote}
 Das $sigma$ steht für die Abzählbarkeit der Vereinigung. Für bestimmte Modellierungen ist die Potenzmenge zu groß, daher beschränkt man sich auf eine Teilmenge.
\end{quote}

\begin{itemize}
 \item $\Omega, \emptyset \in \Sigma$
 \item $A \in \Sigma \Rightarrow A^c \in \Sigma$
 \item Für eine Folge $A_1,A_2,... \in \Sigma$ gilt:
$$ \bigcup_{i=1}^\infty A_i \in \Sigma$$ ( $\Sigma$-Abgeschlossenheit )
\end{itemize}
Ereignisse über einer Menge $\Omega$ werden in der Wahrscheinlichkeitstheorie durch eine geeignete $\sigma$-Algebra beschrieben wird.

Was ist mit $A \cap B$ falls $A,B \in \Sigma$? Ist $A \cap B \in \Sigma$?
$$ A \cap B = (A^c \cup B^c)^c$$
$$ \Rightarrow A^c, B^c \in \Sigma \Rightarrow A^c \cup B^c \in \Sigma \Rightarrow (A^c \cup B^c)^c \in \Sigma $$

$\{\emptyset, \Omega\}$ ist die kleinste und $\mathcal P(\Omega)$ ist die größte $\sigma$-Algebra über $\Omega$.

