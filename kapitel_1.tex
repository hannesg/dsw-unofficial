
\chapter{Mathematische\ Modellierung\ des\ Zufalls}
\section{Zufallsexperimente}
\subsection{Das Würfeln}
\Vorlesung{14.4.2010}
$\Omega = \lbrace 1, ... , n  \rbrace , n \in \mathbb{N}$ Wir möchten zufällig genau eine Zahl aus $\Omega$ ziehen. Eine Möglichkeit: n-seitiger Würfel\newline
Ansatz: $Pr[i] = \frac{1}{n}  \forall i \in \Omega$\newline
$Pr \mathrel{\widehat{=}} ``Probability''$\newline
Sei $A=\lbrace a_1, ... , a_k \rbrace \subset \Omega$, dann ist $Pr[A]=\frac{\left|A\right|}{n} = \frac{k}{n}$ die Wahrscheinlichkeit, dass $a_1, ..., a_{k-1}$ oder $a_{k}$ ausgewählt werden.\newline
A nennt man \Begriff{Ergeignis}\newline
Wenn alle $Pr[i]$ gleich sind so spricht man von einer \Begriff{Gleichverteilung}.\newline
Bei Spielen: n=6
Es herrscht Unabhängigkeit der Würfe, d.h. Ergebnisse beeinflussen sich nicht.

\subsection{Das Lottospiel}
Es werden 6 Zahlen aus 49 gezogen, sagen wir $a_1, .. a_6$. Wir nehmen an, dass wir diese schon geordnet haben: $a_1 < ... < a_6$. Eine Ziehung ist ein Vektor $(a_1,...,a_6)$ mit $a_1 < ... < a_6$. Ergebnisse sind diese Vektoren. Man fasst die Ergebnisse zu einem \Begriff{Grundraum} zusammen, den wir üblicherweise $\Omega$ nennen.
$$ \Omega = \lbrace \lbrace a_1,...,a_6 \rbrace \mid a_i \in \lbrace 1,...,49 \rbrace \forall i=1...6  \rbrace$$
$$ \left|\Omega\right| = \binom{49}{6} = 13983816$$%
Wie hoch ist die Wahrscheinlichkeit, dass ein Tipp 6 Richtige hat? Allgemeiner: k Richtige? %
Welche $\lbrace a_1,...,a_6 \rbrace \in \Omega$ haben $k$ Stellen gemeinsam mit dem Tipp? %
Die bezeichnen wir als \Begriff{günstige\ Ereignisse}.\newpage%

$$A = \lbrace \lbrace a_1,...,a_6 \rbrace \in \Omega \mid \left| \lbrace a_1,...,a_6 \rbrace \bigcup \lbrace b_1,...,b_6 \rbrace \right| = k \rbrace$$%

Wenn ein Element aus $A$ gezogen wird haben wir $k$ Richtige.%
$$ Pr[k \textrm{Richtige}]=\frac{\left| A \right|}{\left| \Omega \right|} = \frac{\binom{6}{k} \cdot \binom{43}{6-k}}{\binom{49}{6}}$$
$$\Rightarrow Pr[6 \textrm{Richtige}]= \frac{\binom{6}{6} \binom{43}{0}}{\binom{49}{6}}=\binom{1}{49}$$
\Vorlesung{19.4.2010}
\subsection{Das Geburtstagsproblem}
Wie hoch ist die Wahrscheinlichkeit, dass zwei Personen in einer Gruppe von $m$ Personen am gleichen Tag Geburtstag haben?
z.B.: $m=50$
% Abb 1
\subsection{Das Ziegenproblem}
Bei einem Fernsehquiz gibt es drei Türen, wobei hinter zwei Türen eine Ziege steht und hinter der dritten Tür ein Hauptgewinn. %
Der Showmaster fragt den Kandidaten, welche Tür er öffnen soll. Nachdem der Kandidat einen Tipp abgegeben hat, bietet der Showmaster seine Hilfe an: %
Er schlägt vor eine Tür zu öffnen und danach soll der Kandidat die Möglichkeit haben seinen Tipp zu verändern.%
Würden Sie sein Angebot annehmen?
% Abb 2
\begin{quote}
 In diesem Fall ist es nicht so einfach, da die günstigen Ergeignisse nicht unabhängig sind.
\end{quote}
\section{Algebra und Maße}
Die axiomatische Fundierung der Wahrscheinlichkeitsrechnung wurder von Kolmogoroff Anfang des 20. Jahrhunderts gegeben. Erst dadurch wurde Wahrscheinlichkeitsrechnung eine wissenschaftliche und quantitative Disziplin.
\subsection{Die Axiome von Komolgoroff}
\begin{itemize}
 \item Ergebnisse eines Zufallsexperimentes fassen wir in einer Menge $\Omega$ zusammen.
 \item \Begriff{Ereignisse} sind gewisse Teilmengen von $\Omega$.
 \item Wahrscheinlichkeiten werden auf Ereignisse bezogen und sind Zahlen zwischen 0 und 1. Eine Zahl $p \in [0,1]$,z.B. $p=\frac{3}{4}$ besagt dass das entsprechende Ereigniss in $\frac{3}{4}$ der Zufallsexperimente auftrit, oder die Wahrscheinlichkeit ist 75 Prozent.

$p=0$: Das Ereignis tritt nicht auf.

$p=1$: Das Ereignis tritt stets auf.
\end{itemize}
Beispiel: Wir werfen eine faire Münze. Kopf und Zahl kommen mit der gleichen Wahrscheinlichkeit.
$$ \textrm{Menge } \Omega = \{\textrm{Kopf},\textrm{Zahl}\} = \{0,1\} $$
$$ \textrm{Ereignisse} = \{\{0\},\{1\},\emptyset,\{0,1\}\} = \mathcal P(\Omega) = \textrm{Potenzmenge von } \Omega $$
$$ P(\{0\})=\frac{1}{2}=P(\{1\}), P(\emptyset)=0 , P(\{0,1\})=1$$
\subsection{``Erweiterung'' der Kolmogoroff-Axiome}

Wie muss die Menge der Ereignisse aussehen?
\begin{itemize}
 \item $A$ Ereignis mit Wahrscheinlichkeit $p$. Wir möchten das Gegenereignis (``$A$ tritt nicht auf'' = $A^c$) ebenfalls als Ereignis ansehen mit der Wahrscheinlichkeit $1-p$.
 \item Ereignisse $A$,$B$ mit den Wahrscheinlichkeiten $p$ und $q$ . Was ist das Ereignis ``$A$ oder $B$'' bzw. ?

 ``$A$ oder $B$'' = $A \cup B$

 ``$A$ und $B$'' = $A \cap B$

 Mithin sollten auch diese Ereignisse in der Ereignismenge drin sein.
 \item Was ist $P(A \cup B)$?

Situation: $A \cap B = \emptyset$ dann ist $P(A \cup B) = P(A) + P(B)$

\end{itemize}
\subsection{$\sigma$-Algebra}
\begin{definition}
Es sei $\Omega$ eine Menge und $\Sigma \subseteq \mathcal P(\Omega)$. $\Sigma$ heißt $\sigma$-Algebraüber $\Omega$ fallse folgende Eigenschaften gelten:
\end{definition}
\begin{quote}
 Das $\sigma$ steht für die Abzählbarkeit der Vereinigung. Für bestimmte Modellierungen ist die Potenzmenge zu groß, daher beschränkt man sich auf eine Teilmenge.
\end{quote}

\begin{itemize}
 \item $\Omega, \emptyset \in \Sigma$
 \item $A \in \Sigma \Rightarrow A^c \in \Sigma$
 \item Für eine Folge $A_1,A_2,... \in \Sigma$ gilt:
$$ \bigcup_{i=1}^\infty A_i \in \Sigma$$ ( $\Sigma$-Abgeschlossenheit )
\end{itemize}
Ereignisse über einer Menge $\Omega$ werden in der Wahrscheinlichkeitstheorie durch eine geeignete $\sigma$-Algebra beschrieben wird.

Was ist mit $A \cap B$ falls $A,B \in \Sigma$? Ist $A \cap B \in \Sigma$?
$$ A \cap B = (A^c \cup B^c)^c$$
$$ \Rightarrow A^c, B^c \in \Sigma \Rightarrow A^c \cup B^c \in \Sigma \Rightarrow (A^c \cup B^c)^c \in \Sigma $$

$\{\emptyset, \Omega\}$ ist die kleinste und $\mathcal P(\Omega)$ ist die größte $\sigma$-Algebra über $\Omega$.
\Vorlesung{21.4.2010}
\begin{example}
Sei $\mathcal F$ die Menge der abgeschlossenen Intervalle der Form $[a,b] \in \mathbb R^d; a,b \in \mathbb R^d$
$$ [a,b]=[a_1,b_1] \times ... \times [a_d,b_d], a=(a_1,...,a_d), b=(b_1	,...b_d)$$

Für d=1: $[a,b]$

Für d=2: 

% Rechteck
Für d=3: 

% Würfeln
\end{example}
$\mathcal B^d$ sei die kleinste $\sigma$-Algebra über $\mathbb R^d$, die $\mathcal F$ enthält. Unter kleinste verstehen wir, dass jede $\sigma$-Umgebung $\Sigma$ mit $\Sigma \supset \mathcal F$ und $\Sigma \subseteq \mathcal{B}^d$ $\Sigma = \mathcal{B}^d$ erfüllt.

Warum existiert $\mathcal{B}^d$?
\begin{quote}
 Dann hat es auch keinen weiteren Grund, dass ich diese Vorlesung halte.
\end{quote}
\begin{quote}
 $\mathcal{B}^d$ existiert, da mit $\mathcal P(\mathbb R^d) \supseteq \mathcal F$ eine $\sigma$-Algebra existiert. Somit existiert auch eine minimale $\sigma$-Algebra.
\end{quote}
Sei $M=\{\Sigma | \Sigma ist \sigma-\textrm{Algebra über} \mathbb R^d, \mathcal F \subseteq \Sigma\}$.
Suche die kleinste $\sigma$-Algebra aus $M$. Das geht wenn $M \neq \emptyset$. Da $\mathcal P(\mathbb R^d) \in M$ ist dies gegeben.
Definition 1.2 $\mathcal B^d$ nennt man die \Begriff{Borelsche $\sigma$-Algebra} über $\mathbb{R}^d$.
Wie sehen die Mengen in $\mathcal B^d$ aus?

Für eine Menge $X \subseteq \mathcal P(\Omega)$ verstehen wir unter $\sigma(X)$ die kleinster $\sigma$-Algebra, die $X$ enthält. Sie $\mathcal O$ die Menge der offenen Intervall in $\mathbb R^d$. Betrachte $\sigma(\mathcal O)$.

\begin{theorem}
$$ X \subseteq \mathcal P(\Omega); Y \subseteq \mathcal P(\Omega), X \subseteq Y \Rightarrow$$
$$ \sigma(X) \subseteq \sigma(Y)$$
\end{theorem}
\begin{proof}
Nehme $Z \in \sigma(X) \Rightarrow Z \in \sigma(Y)$.

Ausgangssituation:
$\Omega$ ist gegeben. $X$ ist Menge von Teilmengen von $\Omega$ d.h. $X \subseteq \mathcal P(\Omega)$.
Dann ist $\sigma(X)$ die kleinste $\sigma$-Algebra über $\Omega$, die $X$ enthält, d.h. $X \in \sigma(X)$.
\end{proof}

\Vorlesung{26.4.2010}
$\sigma(\mathcal O)$ ist die kleinste $\sigma$-Algebra, die die Menge $\mathcal O$ der offenen Intervalle in $\mathbb R^d$ enthält.

\begin{theorem}
$\mathcal B^d = \sigma(\mathcal O)$
\end{theorem}

\begin{proof}
Zur Vereinfachung sei $d=1$.
\begin{itemize}
 \item[Zunächst $\mathcal B^d \subseteq \sigma(\mathcal O)$:]
Wir wissen, dass $\mathcal B^d = \sigma(\mathcal F)$ ist. $\mathcal F$ ist Erzeuger von $\mathcal B^d$.
\begin{quote}
 Bei der $\sigma$-Algebra reicht es zu Zeigen, dass der Erzeuger in der Menge enthalten ist.
\end{quote}
Falls $\mathcal F \subseteq^{(*)} \sigma(\mathcal O)$ dann gilt $\sigma( \mathcal F) \subseteq \sigma( \sigma( \mathcal O)) = \sigma(\mathcal(O))$.
\begin{quote}
 $\sigma(\sigma(M)) = \sigma(M)$ da $\sigma (M)$ bereits eine $\sigma$-Algebra ist.
\end{quote}
Das heißt: es reicht $(*)$ zu zeigen. Sei $[a,b] in \mathcal F$ ein beliebiges geschlossenes Interval. Wie können wir $[a,b]$ mit Hilfe von offenen Intervallen oder anderen Elementen aus $\sigma(\mathcal O)$ erzeugen?

Vorschlag:
$$ [a,b] = ( ]-\infty, a[ \cup ]b,\infty[ )^c$$
\begin{quote}
 Schon nicht schlecht, aber die Frage ist, ob $]-\infty, a[ \in \sigma(\mathcal O)$ ist. Denn bisher haben wir die offenen Intervalle die Form $]x,y[; x,y \in \mathbb R$ und $\infty \notin \mathbb R$.
\end{quote}

Vorschlag2:
$$ \sigma(\mathcal O) \ni \bigcup_{n=1}^\infty ]b,b+n[ = ]b,\infty[ $$
\begin{quote}
 Da dies eine abzählbare Vereinigung ist von Mengen aus $\sigma(\mathcal O)$ ist auch die Vereinigung wieder aus $\sigma(\mathcal O)$. Somit ist der Vorschlag brauchbar.
\end{quote}
Folglich ist jedes Element auf $\mathcal F$ auch in $\sigma(\mathcal O)$.
\item[Nun $\sigma(\mathcal O) \subseteq \mathcal B^d$:]
Es reicht zu zeigen, dass $\mathcal O \subseteq \mathcal B^d$ ( wie eben ). Sei $]a,b[ \in \mathcal O$ beliebig und $a,b \in \mathbb R$. Zu zeigen $]a,b[ \in \mathcal B^d$.

Vorschlag:
$$ \mathcal B^d \ni \bigcup_{n=1}^\infty [a+\frac{1}{n},b-\frac{1}{n}] = ]a,b[ $$
\begin{quote}
 Wieder handelt es sich um eine abzählbare Vereinigung von Elementen aus $\mathcal B^d$. Die Grenzen $a$ und $b$ sind natürlich nicht in der Vereinigung, aber man kann für jede Zahl $c$ zwischen $a$ und $b$ ein $n$ finden sodass $c$ in der Vereinigung ist.
\end{quote}
\end{itemize}

Wir wissen:

$$ \mathcal B ^d \subseteq \mathcal P (\mathbb R^d) $$
Frage:
Gibt es Mengen $A \subseteq \mathbb R^d$ mit $A \neq \mathcal B^d$? bzw. $ \mathcal B ^d = \mathcal P (\mathbb R^d) $?

Vorschlag:
$$ \mathbb R^d \supset A = \bigcup_{a\in A} [a] \in \mathcal B^d $$
Die Vereinigung ist nur in $\mathcal B^d$ falls $A$ abzählbar ist.

Es gibt Teilmengen des $\mathbb R^d$ die keine Borelmengen sind!
$$ \mathcal B ^d \subsetneq \mathcal P (\mathbb R^d) $$
Diese nennt man \Begriff{Cantor-Mengen}.
\end{proof}


Um die Wahrscheinlichkeit eines Ereignisses zu definieren möchten wir - wenn Ereignisse durch Elemente einer $\sigma$-Algebra beschreiben werde, also Teilmengen einer Grundmenge $\Omega$ sind - diesen Teilmengen eine Maßzahl zuordnen, d.h. sie messen.
\begin{itemize}
 \item[1.Fall]
$$ \lvert \Omega \rvert < \infty, A \subseteq \Omega$$
$\lvert A \rvert$ ist ein Maß. Bei Gleichverteilung ergibt $\frac{\lvert A \rvert}{\lvert \Omega \rvert}$.
 \item[2.Fall]
$$ \lvert \Omega \rvert = \infty, A \subseteq \Omega$$
Falls $\lvert A \rvert = \infty$ so ist ein Maß der Form $\lvert A \rvert$ zu undifferenziert.
\end{itemize}
\section{Wahrscheinlichkeitsraum}
Ist $\Sigma$ eine $\sigma$-Algebra über $\Omega$, so nennt man $(\Omega,\Sigma)$ einen \Begriff{Meßraum}.

\begin{definition} Sei $(\Omega,\Sigma)$ ein Meßraum.
\begin{enumerate}
 \item Eine Funktion $µ: \Sigma \rightarrow [0,\infty]$ heißt \Begriff{Maß} falls folgendes gilt:
  \begin{enumerate}
   \item $µ(\emptyset) = 0$

   \item für jede Folge von paarweise disjunkten Mengen $A_1, A_2, ... \in \Sigma$ gilt:
$$ µ \left ( \bigcup_{i=1}^\infty A_i \right ) = \sum_{i=1}^\infty µ \left ( A_i \right ) $$
  \end{enumerate}
 \item Ein Maß $P: \Sigma \rightarrow [0,1]$ mit $P(\Omega) = 1$ heißt \Begriff{Wahrscheinlichkeitsmaß}.
 \item Das Tripel $(\Omega,\Sigma,P)$ nennt man \Begriff{Wahrscheinlichkeitsraum}
\end{enumerate}
\end{definition}

Zusammenhang zur ``experimentellen'' oder ``intuitiven'' Wahrscheinlichkeit:

\subsection{Komolgoroff-Axiome}
Ein Zufallsexperiment wird im mathematischen Sinne durch einen Wahrscheinlichkeitsraum $(\Omega,\Sigma,P)$ beschrieben. Dabei gilt:
\begin{enumerate}
 \item $\Omega$ umfasst alle Ergebnisse $\omega \in \Omega$ des Zufallsexperimentes.
 \item Die Ereignisse sind die Mengen aus $\Sigma$.
 \item Für eine Menge $A \in \Sigma$ ist $P(A)$ die Wahrscheinlichkeit des Ereignisses $A$.
\end{enumerate}

\Vorlesung{28.4.2010}
\subsection{Eigenschaften eines Wahrscheinlichkeitsmaßes}
\begin{proposition}
 Sei $(\Omega,\Sigma,P)$ ein Wahrscheinlichkeitsraum. Es seien $A,B,A_1,A_2, ... \in \Sigma$. Dann gilt:
 \begin{enumerate}
  \item $P(A^c) = 1 - P(A)$
  \item $P( A \cup B ) + P( A \cap B ) = P(A) + P(B)$
  \item Aus $A \subseteq B$ folgt $P(B \setminus A) = P(B) - P(A)$
  \item Aus $A \subseteq B$ folgt $P(A) \leq P(B)$
  \item Für $A_1, ... A_n \in \Sigma$ gilt:
$$ P \left ( \bigcup_{i=1}^n A_i \right ) \leq \sum_{i=1} P(A_i)$$ ( Union-Bound )
 \end{enumerate}
\end{proposition}
\begin{proof}

 \begin{enumerate}
  \item 
\begin{quote}
 Wir wissen schon, dass $P(\Omega) = 1$ gilt. Folglich ist zu zeigen, dass $ P(A^c) + P(A) = 1 = P(\Omega)$ gilt.
\end{quote}
\begin{equation}
 P(\Omega)
 = P(A \cup A^c)
 = P(A) + P(A^c)
\end{equation}
\begin{quote}
 Der letzte Schritt ist möglich, da $A$ und $A^c$ paarweise disjunkt sind.
\end{quote}
  \item
$$ A \cup B = A \cup ( B \setminus A ) $$%
Daraus folgt aufgrund der $\sigma$-Additivität:%
$$ P(A \cup B) = P(A) + P( B \setminus A ) $$%
Weiterhin gilt:%
$$ B = B \setminus A \cup (A \cap B) \Rightarrow P(B) = P(B \setminus A) + P(A \cap B)$$
$$ \Rightarrow P(B \setminus A) = P(B) - P(A \cap B) $$
$$ \Rightarrow P(A \cup B) = P(A) + P(B) - P(A \cap B) $$
  \item
$$ B = A \dot{\cup} (B \setminus A) $$
$$ \Rightarrow P(B) = P(A) + P(B \setminus A) $$
$$ \Rightarrow P(B \setminus A) =  P(B) - P(A) $$
  \item 
$$ B = A \dot{\cup} (B \setminus A) $$
$$ \Rightarrow P(B) = P(A) + P(B \setminus A) \geq P(A) $$
  \item
\begin{quote}
 Für $n=2$ ist die Behauptung schon gezeigt.
\end{quote}
Versuch Darstellung als:
$$ \bigcup_{i=1}^n A_i =  \bigcup_{i=1}^l B_i $$
mit $B_i$ paarweise disjunkt.
$$ B_1 = A_1 $$
$$ B_2 = A_2 \setminus A_1 $$
$$ B_3 = A_3 \setminus ( A_1 \cup A_2) $$
$$ B_n = A_n \setminus ( A_1 \cup ... \cup A_{n-1} ) $$
Die $B_i$ sind immer paarweise disjunkt und $B_i \subseteq A_i ; i \in 1,...,n$ so wie
$$ \bigcup_{i=1}^n A_i =  \bigcup_{i=1}^n B_i $$
$$ \Rightarrow P \left ( \bigcup_{i=1}^n A_i \right ) = P \left ( \bigcup_{i=1}^n B_i \right ) = \sum_{i=1}^n P(B_i) $$
Nun wissen wir aber nach Konstruktion, dass jedes $B_i \subseteq A_i$ und somit $P(B_i) \leq P(A_i)$.
Daraus folgt:
$$ \sum_{i=1}^n P(B_i) \leq \sum_{i=1}^n P(A_i) $$
 \end{enumerate}
\end{proof}

\subsection{Der Laplace-Raum}
Sei $\Omega$ endllich. Sei $\Sigma = \mathcal P(\Omega)$ und $P: \Sigma \rightarrow [0,1], P(A):= \frac{\lvert A \rvert}{\lvert \Omega \rvert}$ für alle $A \in \Omega$. %
$(\Omega, \Sigma)$ ist ein Meßraum, weil $\Sigma = \mathcal P(\Omega)$ eine $\sigma$-Algebra.
$P$ ist ein Wahrscheinlichkeitsmaß, da:
\begin{enumerate}
 \item $P(A) \in [0,1] \forall A \subseteq \Omega$
 \item $P(\emptyset) = \frac{\lvert A \rvert}{\lvert \Omega \rvert} = \frac{0}{\lvert \Omega \rvert} = 0$
 \item $P(\Omega) = \frac{\lvert \Omega \rvert}{\lvert \Omega \rvert} = \frac{\lvert \Omega \rvert}{\lvert \Omega \rvert} = 1$
 \item $A_1,A_2,... A_n \subseteq \Omega$ paarweise disjunkt.
$$ P(A_1 \cup ... \cup A_n) = \frac{\lvert A_1 \cup ... \cup A_n \rvert}{\lvert \Omega \rvert} = \frac{\lvert A_1 \rvert + ... \lvert A_n \rvert}{\lvert \Omega \rvert} = $$
$$ \frac{\lvert A_1 \rvert}{\lvert \Omega \rvert} + ... +  \frac{\lvert A_n \rvert}{\lvert \Omega \rvert} = P(A_1) + ... + P(A_n) = \sum_{i=1}^n P(A_i) $$
\begin{quote}
 Da $\Omega$ endlich ist, gibt es auch nur endlich viele verschieden Teilmengen.
\end{quote}

Mit ``Übergang'' aus $n \rightarrow \infty$ ist $P$ $\sigma$-Additiv, also ein Wahrscheinlichkeitsmaß.
\end{enumerate}
Also ist $(\Omega,\Sigma,P)$ ein Wahrscheinlichkeitsraum.


\begin{enumerate}
 \item 
Gibt es ein Wahrscheinlichkeitsmaß $P$ auf $\mathbb N$ mit $\mathcal P(\mathbb N)$ als $\sigma$-Algebra, so dass $P(\{i\}) > 0 \forall i \in \mathbb N$?
 \item
Gibt es ein Wahrscheinlichkeitsmaß $P$ auf $\mathbb R$ mit geeigneter $\sigma$-Algebra, die alle einelementigen Mengen $\{x\}; x \in \mathbb R$ enthält und %
$P(\{x\}) > 0$?
\end{enumerate}
\paragraph*{}
\label{Zu 1}
$P({i}) = \frac{1}{2^i}; i = 1,2,... $ (geometrische Folge). Dann ist
$$P(\mathbb N) = P \left ( \bigcup_{i=1}^\infty \{i\} \right ) = \sum_{i=1}^\infty \frac{1}{2^i} %
= \frac{1}{1-\frac{1}{2}} - 1 = 1$$ 
